\chapter{Methodology}

This chapter describes the methodology of the evaluation as well as its application. It can be quite difficult to put together a good structure for a paper or a study that details some complicated research especially in a highly specialized field. With this in mind, we decided to opt for a framework proposed by Victor Basili of the University of Maryland, College Park and the Software Engineering Laboratory at the NASA Goddard Space Flight Center\todo{citation needed} called \textit{Goal Question Metric} (or just GQM).

\section{The Goal Questions Metric Approach}

The focus of the GQM approach is to define a good measurement mechanism mainly for engineering disciplines like software engineering, computer science and others. Application of this approach helps support project planning, determine pros and cons of the current project and process, provide an intuition about the impact of modifying or refining a technique as well as assess current progress and last but not least to write the final work in a comprehensible and structured way.

The first prominent step in terms of GQM is to define a set of goals. Goals are entities that are to be assessed and therefore must be defined in a way that allows their assessment. Second step is to define a set of questions, these characterize the way the assessment of a goal is going to be carried out. The last step at the very bottom is to define metrics that are used to answer the questions in a quantitative way. the usual workflow is to define goals and then refine each into several separate questions. The questions are then further refined into metrics where more than one question can have the same metric in common.

\section{Application of GQM}

Goal: \textbf{Analyze} machine learning models \textbf{in order to} find the best possible approach \textbf{with respect to} assignee prediction \textbf{from the point of view of} project managers \textbf{in the context of} issue tracking systems.