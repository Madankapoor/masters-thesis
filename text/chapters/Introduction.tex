\chapter{Introduction}

The advancements at the end of 20th and the beginning of 21st century brought many improvements in \textit{Machine Learning} (ML) and \textit{Natural Language Processing} (NLP) algorithms allowing us to predict and determine the correct action in many domains based solely on prior knowledge with almost no human interaction~\cite{carbonell1983overview}. The possible applications of these techniques are very broad ranging from medicine to transportation, engineering, research and many more~\cite{kononenko2001machine}\cite{nguyen1990neural}. In this thesis, we take advantage of ML and NLP in the context of empirical software engineering. Specifically, we focus on automatic bug triage---that is the assignment of a proper developer for bug resolution.

With the advent of computer software in our contemporary world, it was quickly discovered that software is almost never perfect and it needs to be continuously maintained as new issues (most commonly \textit{bugs}---errors in computer programming that cause unexpected behavior, but can also be an enhancement, feature request etc.) keep arising as long as the computer program is used~\cite{nist2002}. It is not unusual to discover thousands or even tens of thousands of bugs in a software application and therefore there are usually more than one developer working on these bugs in order to fix them. For convenience and to reduce the effort necessary to maintain the list of the issues, the software development community came up with web-based applications designed specifically to track them~\cite{bertram2010communication}. This kind of software is called an \textit{issue tracker} and one entry of an issue is called a \textit{bug report} or \textit{ticket}. A bug report usually contains the summary, description and \textit{assignee} of the discovered bug, as well as other fields (priority, status, comments etc.). Assignee is the developer who is assigned to investigate and possibly fix the bug. The necessity of determining the assignee raises an important question---who should fix the bug?~\cite{Anvik2006} The goal of this thesis is to find a way to answer this question in real time with nearly no human interaction.

The initiative for this study arose from a Czech-based company (that wished to remain anonymous) which requested an automated bug assignment application it could use to speed up its internal workflow. Therefore, besides this work, we created a web application based on the findings from this thesis. The link to the source code of the application can be found in appendix~\ref{appendix:links}.

\section{Problem Statement}

The workflow used for issue resolution should be as time-effective as possible. In an ideal scenario, when a new bug is reported, it is immediately assigned to the most relevant and the least occupied developer based on the priority of the new bug report.

\textit{However}, current solutions heavily rely on a party (usually a combination of project management and software engineers) that evaluates the importance of each bug report and attempts to assign them to the relevant developers (commonly referred to as \textit{triage}). This solution creates an unnecessary overhead that takes up resources (man-hours) which could be used more efficiently.

The approach we chose to resolve the problem is the automation of bug assignment with computers. Our solution uses ML algorithms to predict the correct assignee based on previous bug reports. The benefit is that it does not require any human input to make its decision. The disadvantage is that it neither takes into account the priority of the reported bug, nor the workload of the developer who is picked for the assignment.

\section{Objectives}

The first objective is to study related work to determine the baseline for our own experiments and to pick the best candidate models. Reviewing related work will also allow us to eventually compare our results with previous attempts.

The primary objective of this thesis is to evaluate the chosen ML models and to determine which one with what configuration is the best choice for this problem considering all its aspects. The most important aspect is the performance, which is the quality of the predictions done by the model based on several metrics that will be detailed later in the thesis.

Machine Learning approaches are very often dependent on the quality of the datasets to which models are applied. One of the objectives of the thesis is to analyze the chosen datasets, including a proprietary dataset provided by a Czech-based company, to determine whether the datasets have similar properties. This will allow us to conclude whether our approach can be generalized for all project repositories.

As we were able to retrieve data from open-source projects (accessible on the Internet) as well as from a proprietary project, our last objective is to analyze and compare the performance of the models on these datasets to determine whether there is a general difference between them.

\section{Outline}

In the second chapter, we summarize some related work from various scientific journals fulfilling our first objective. Next, we introduce ML and text classification including the background of the models and techniques we use in our evaluation. In the third chapter, we establish the methodology of our work. The fourth chapter is the core of the thesis, it includes the evaluation of the models and analysis of the datasets. In the second to last chapter, we discuss the results, compare them to related work and consider some possible threats to validity. The last chapter concludes this study with a summary of our findings and an outline for possible future work.
