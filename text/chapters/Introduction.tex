\chapter{Introduction}

The advancements at the end of 20th and the beginning of 21st century brought many improvements in \textit{Machine Learning} (ML) and \textit{Natural Language Processing} (NLP) algorithms allowing us to predict and determine the correct action based solely on prior knowledge with almost no human interaction~\cite{carbonell1983overview}. The possible applications of these techniques are very broad ranging from medicine to transportation, engineering, research and many more~\cite{kononenko2001machine}\cite{nguyen1990neural}. In this thesis, we take advantage ML and NLP in the context of empirical software engineering. Specifically, we focus on automatic bug triage -- that is the assignment of a proper developer for bug resolution.

With the advent of computer software in our contemporary world, it was quickly discovered that software is almost never perfect and it needs to be continuously maintained as new issues (most commonly \textit{bugs} -- errors in computer programming that causes unexpected behavior, but can also be an enhancement, feature request etc.) keep arising as long as the computer program is used~\cite{nist2002}. It is not unusual to discover thousands or even tens of thousands of bugs in a software application and therefore there are usually more than one developer working on these bugs in order to fix them. For convenience and to reduce the effort necessary to maintain the list of the issues, software development community came up with web-based application designed specifically to track them~\cite{bertram2010communication}. This software is called an \textit{issue tracker} and one entry of an issue is called a \textit{bug report} or \textit{ticket}. A bug report usually contains the summary, description and \textit{assignee} of the discovered bug, as well as other fields (priority, status, comments etc.). Assignee is the developer who is assigned to investigate and possibly fix the bug. The necessity of determining the assignee raises an important question -- who should fix the bug?~\cite{Anvik2006} The goal of this thesis is to find a way to answer this question in real time with nearly no human interaction.

\section{Problem Statement}

In this section, we formally present what the problem that our works addresses is, who is the party that is concerned with its resolution and what form our resolution takes.

The problem we try to address is, as introduced above, to reduce the effort needed to determine the correct developer to a newly reported bug report. The amount of work needed to correctly assign developers to bugs increases significantly with higher turnover of bug reports as well as higher number of developers working on the project. This spawns the initiative to discover a solution that effectively cuts down on the man-hours needed to address this concern.

The party most concerned with this problem is usually someone in charge of project management. This depends a lot on the internal structure of the project organization, but there is usually someone that is assigning work to other developers which fulfills this role. On the other hand, at least a part of this burden can be addressed either by the reporters (especially if they are also developers who work on the project as well) or by developers themselves (most often quality assurance engineers).

The approach we chose to resolve the problem is the automation of bug assignment with computers. Our solution uses ML algorithms to predict the correct assignee based on previous bug reports. This has some limitations, however. There must be a large enough datasets of bug reports in the repository already (at least a couple of thousands) and the time and hardware resources required for realization are quite substantial. Despite these drawback, we opt for this resolution as it is the only approach that offers performance apt for viable utility.

\section{Objectives}

The first objective is to study related work to determine the baseline for our own experiments and to pick the best candidate models. Reviewing related work will also allow us to eventually compare our results with previous attempts.

The primary objective of this thesis is to evaluate different machine learning models and to determine which one with what configuration is the best choice for this problem considering all its aspects. Most important aspect of all is the performance, which is the quality of the predictions done by the model based on several metrics that will be detailed later in the thesis.

Machine Learning approaches are very often dependent on the quality of the datasets to which models are applied. One of the objectives of the thesis is to analyze the chosen datasets, including a proprietary dataset provided by a Czech-based company, to determine the affinity of the datasets. This will allow us to conclude whether our approach can be generalized for all project repositories.

As we were able to retrieve data from open-source projects (accessible on the Internet) as well as from a proprietary project, our last objective is to compare the performance of these datasets. 

\section{Outline}

The first chapter is an introduction to Machine Learning as wall as the problem domain of this thesis and the summary of its objectives. In the second chapter, we summarize some related work from various scientific journals fulfilling our first objective. Next, we introduce ML and text classification including the background of the models and techniques we use in our evaluation. In the third chapter, we establish the methodology of our work. The fourth chapter is the core of the thesis, it includes the evaluation of the models and analysis of the datasets. In the second to last chapter, we discuss the results, compare them to related work and consider some possible threats to validity. The last chapter concludes this study with a summary of our findings and an outline for possible future work.